
%This is the hello world document, it is very usefull
%As you can see the character % is used for comments

\documentclass[a4paper,12pt]{article}
%\documentclass[doc]{apa}
\usepackage[usenames,dvipsnames]{color}
\usepackage{graphicx}
\usepackage{multicol}
\usepackage{multirow}
\usepackage{amsmath}
\usepackage[T1]{fontenc}
\usepackage[debug]{hyperref}
\setlength\parindent{0pt}
%\usepackage{figure}
\usepackage[margin=0.5in]{geometry}

\begin{document}
\small
Some info (Coastal exposure\dots):
\begin{figure}[h]
    \centering
    \includegraphics[totalheight=.10\textheight]{1.png}
    \caption{ }
    \label{fig:mobile1}
\end{figure}


The computation of the drif is given by:

\begin{equation}
\begin{split}
\frac{dx_i}{dt} = u_{sea}(x_i, t) + C_D D u_{wind}(x_i,t) + C_W u_{wave} + R \sqrt{\frac{2K_h}{r dt}} 
\end{split}
\end{equation}

Merging waves and wind: \\
\begin{equation}
\begin{split}
\frac{dx_i}{dt} = u_{sea}(x_i, t) + C_D D u_{wind}(x_i,t) + R \sqrt{\frac{2K_h}{r dt}} 
\end{split}
\end{equation}

$C_D$ Wind drag coefficient. \\
$D$ Rotation matrix (wind deflection).\\

Merging rotation and wind coefficient
\begin{equation}
\begin{split}
\frac{dx_i}{dt} = u_{sea}(x_i, t) + \pmb{A}u_{wind}(x_i,t) + R \sqrt{\frac{2K_h}{r dt}} 
\end{split}
\end{equation}

$\pmb{A}$ obtained by the paper for open waters : 

\begin{equation}
\pmb{A} = 
    \begin{bmatrix}
        0.81 & -0.13 \\
        0.06 & 0.76 \\
    \end{bmatrix}
\end{equation}

\textbf{Pereiro formula for open waters:}
\begin{equation}
\begin{split}
\frac{dx_i}{dt} = u_{sea}(x_i, t) + \alpha(\theta) u_{wind}(x_i,t) + R \sqrt{\frac{2K_h}{r dt}} 
\end{split}
\end{equation}

$K_h  = 1 \frac{m^2}{s}$ Turbulent horizontal diffusion coefficient. \\
$r  = 1$ Not completely sure, it suppose is related with the standard deviation of the diffusion. \\
$R(x)$ Random number between -1 and 1 \\
\begin{equation}
\alpha(\theta)
 = 
    \begin{bmatrix}
        0.81 \sin(\theta) & -0.13\cos(\theta) \\
        0.06 \cos(\theta) & 0.76 \sin(\theta) \\
    \end{bmatrix}
\end{equation}
Where $\theta$ is the angle of the wind vector (measured clockwise from north).\\

\textbf{Our formula:}
\begin{equation}
\begin{split}
\frac{dx_i}{dt} = u_{sea}(x_i, t) + .025 \alpha(\theta) u_{wind}(x_i,t) + R \sqrt{\frac{2K_h (\frac{m^2}{s})}{r dt (s)}} \\
\frac{dx_i}{dt} = u_{sea}(x_i, t) + .025 \alpha(\theta) u_{wind}(x_i,t) + R \sqrt{\frac{2(1) (\frac{m^2}{s})}{1 dt (s)}} \\
\frac{dx_i}{dt} = u_{sea}(x_i, t) + .025 \alpha(\theta) u_{wind}(x_i,t) + R \sqrt{\frac{2K_h dt (\frac{m^2}{s})}{(1/3) }} 
\end{split}
\end{equation}

$\theta = 15$ Wind deflection angle. Rotation matrix:  
$
\alpha(\theta)
 = 
    \begin{bmatrix}
        \sin(\theta) & -\cos(\theta) \\
        \cos(\theta) & \sin(\theta) \\
    \end{bmatrix}
$

$K_h  = 1 \frac{m^2}{s}$ Turbulent horizontal diffusion coefficient. \\
$r  = 1$ Not completely sure, it suppose to be related with the standard deviation of the diffusion. \\
$R(x)$ Random number between -x and x. \\
$dt = 3600$ Delta t in seconds.\\


\end{document}

